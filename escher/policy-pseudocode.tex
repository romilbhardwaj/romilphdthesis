% Code figure
\newcommand\colmult{0.23}
\begin{figure*}[t]
~
% Task co-location
~
\begin{subfigure}[b]{\colmult \textwidth}
  \centering
  \begin{minted}[fontsize=\tiny,breaksymbolleft=\tiny\ensuremath{},breakautoindent=true]{python}
def task_colocation():
 constraint = sum([t.resources for t in coloc_tasks])
 set_resource("co-grp", len(coloc_tasks), constraint)
 for task in coloc_tasks:
  task.launch(res={"co-grp": 1})
  \end{minted}
  \caption{Task Co-location}
  \label{fig:policycode:taskcoloc}
\end{subfigure}
~
% Data locality TODO: Consider showing task based data locality too.
~
\begin{subfigure}[b]{\colmult\textwidth}
  \centering
  \begin{minted}[fontsize=\tiny,breaksymbolleft=\tiny\ensuremath{},stripnl=false]{python}
def data_locality():
  data_addr = get_location(data)  # Node id
  set_resource("data-loc", 1, data_addr)
  task.launch(res={"data-loc": 1})
  
  
  \end{minted}
  \caption{Data locality}
  \label{fig:policycode:datalocality}
\end{subfigure}
~
% Policy: Anti Affinity
~
\begin{subfigure}[b]{\colmult\textwidth}
  \centering
  \begin{minted}[fontsize=\tiny,breaksymbolleft=\tiny\ensuremath{},stripnl=false]{python}
def anti_affinity():
  for node in nodes:
    set_resource("anti_aff", 1, node)
  for task in anti_affinity_tasks:
    task.launch(res={"anti_aff": 1})

  \end{minted}
  \caption{Anti-affinity}
  \label{fig:policycode:antiaff}
\end{subfigure}
~
% Static Load Balancing
~
\begin{subfigure}[b]{\colmult\textwidth}
  \centering
  \begin{minted}[fontsize=\tiny,breaksymbolleft=\tiny\ensuremath{}]{python}
def static_load_bal():
 cap = ceiling(num_tasks/num_nodes)
 for node in nodes:
  set_resource("load_bal", cap, node)
 for task in tasks:
  task.launch(res={"load_bal": 1})
  \end{minted}
  \caption{Static Load Balancing}
  \label{fig:policycode:staticloadbal}
\end{subfigure}

\newline
\vspace{2.0mm}

~
% Dynamic Load Balancing
~
\begin{subfigure}[b]{\colmult\textwidth}
  \centering
  \begin{minted}[fontsize=\tiny,breaksymbolleft=\tiny\ensuremath{}]{python}
def load_monitor():
 while True:
  res = get_cluster_status()
  if res["load_bal"] == 0 on all nodes:
   set_resource("load_bal", increment 1, all_nodes)
  if res["load_bal"] >= 1 on all nodes:
   set_resource("load_bal", decrement 1, all_nodes)
def dynamic_load_bal():
  task.launch(res={"load_bal": 1})
  \end{minted}
% for task in tasks:
%   task.resources = {"load_bal": 1}
%   task.launch()
  \caption{Dynamic Load Balancing}
  \label{fig:policycode:dynloadbal}
\end{subfigure}
~
% Policy: Binpacking
~
\begin{subfigure}[b]{\colmult\textwidth}
  \centering
  \begin{minted}[fontsize=\tiny,breaksymbolleft=\tiny\ensuremath{},stripnl=false,escapeinside=||,mathescape=true]{python}
def expand_binpack_nodes(res):
 find node where node.resources > res:
  set_resource("binpack", |$\infty$|, node)


def binpacking():
  cluster_res = get_cluster_status()
  if task_res is not subset(binpack_nodes):
    expand_binpack_nodes(task_res)
  task.launch(res={"binpack": 1})
  
  
  \end{minted}
  \caption{Bin-packing}
  \label{fig:policycode:binpacking}
\end{subfigure}
~
% Policy: Gang sched
~
\begin{subfigure}[b]{\colmult\textwidth}
  \centering
  \begin{minted}[fontsize=\tiny,breaksymbolleft=\tiny\ensuremath{}]{python}
def ghost_task(child_tasks, timeout):
  set_resource("reserved", len(child_tasks))
  wait(child_tasks, timeout)
  set_resource("reserved", 0)

def gang_scheduling():
 ghost_task(gang_tasks).launch(res = sum(task_resreqs))
 for t in gang_tasks: 
   t.launch(res={"reserved": 1})
  \end{minted}
  \caption{Gang Scheduling}
  \label{fig:policycode:gangsched}
\end{subfigure}

\caption{Implementation of popular scheduling policies with ephemeral resources. \name{} can allow locality based policies by creating ephemeral resources on nodes which can satisfy the said locality. \name{} can also be used to implement dynamic policies such as load-balancing and bin-packing by implementing lightweight processes which can adjust ephemeral resource capacities on the fly.}
\label{fig:policycode}
\end{figure*}