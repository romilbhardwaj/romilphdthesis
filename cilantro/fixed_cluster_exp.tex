
\subsection{Multi-tenant cluster sharing}
\label{sec:fixedclus}
% \vspace{-0.05in}


% We evaluate Cilantro in a multi-tenant setting where competing jobs from different users share the same cluster. In this environment, we run Cilantro with various online-learning based policies from \S\ref{sec:policies}.
We first evaluate Cilantro's multi-tenant policies (\S~\ref{sec:learningpolicies}) on a 1000 CPU
cluster shared by 20 users.


\textbf{Workloads.}
We use three classes of workloads---database querying, prediction serving and
machine learning training---which are used to create multiple jobs.

The database querying workload runs TPC-DS~\cite{tpc-ds} 
queries on replicated instances of sqlite3 database and uses the query latency as the performance metric. The TPC-DS suite consists of 99 query templates out of which 27 were not compatible with the sqlite dialect and were discarded. The remainder were binned according to their mean latency when measured on a AWS m5.2xlarge instance. The chosen query types and their ids are listed in Table \ref{tab:querybinmap}. From the TPC-DS query set, we created two workloads (setting scale factor to 100): DB-0, which had queries that completed in under 100 ms and DB-1 which had queries that had a completion time between 100 and 300 ms. When a query is requested, we randomly pick a relevant query and dispatch it
according to the trace. The performance metric of interest is  query latency. 

\insertTableQueryBins

In prediction serving, a job processes arriving queries
to output a prediction, usually obtained via a machine learning model.
In our set up, we use a random forest regressor as the model and the the news popularity
dataset~\cite{fernandes2015proactive} for training and test queries in a 50:50 split.
% Half of the dataset is used to train the model (before the experiment),
% while the remaining half (test set) is used to generate queries for the workload.
Queries are picked randomly from the test set and issued in batches of 4.
The metric of interest is the serving latency.

The ML training workload trains a neural network on the naval propulsion~\cite{coraddu2016machine} dataset using stochastic gradient descent. The database querying and prediction serving workloads use the query latency as the performance metric while ML training uses batch throughput to measure performance. Resource-performance mappings for informing the oracle baselines were obtained through offline profiling of all workloads. These profiles are visualized in Figure \ref{fig:profiling}.


\textbf{Multi-tenant cluster jobs setup.} For the multi-tenant cluster resource sharing evaluation, we setup 20 jobs with different workloads and SLOs. Table \ref{tab:expjobsetup} details the exact SLO and utility function for each job.  The utility function for each job is either of \incmtt{linear}, which directly maps performance to utility (Figure \ref{fig:utilityillus}(a)), \incmtt{sqrt}, which performs a sublinear mapping of performance to utility (Figure \ref{fig:utilityillus}(b)), or \incmtt{quadratic}, which performs a superlinear mapping of performance to utility (Figure \ref{fig:utilityillus}(c)).
\insertExpSetupTable

% \newcommand{\wlheader}[1]{\emph{#1}}

% \wlheader{Database querying:}
% We use the TPC-DS \cite{tpc-ds} benchmark suite as the workload backed by replicated instances of sqlite3 database. 
% %Each job's database is
% %populated with the TPC-DS data generator with the scale factor parameter set to 100.
% % The TPC-DS suite consists of 99 query templates
% % out of which 27 were not compatible with the sqlite dialect and were discarded.
% % Of the remaining workloads,
% From the TPC-DS query set, we created two workloads (setting scale factor to 100): DB-0, which had queries that completed in under 100 ms and DB-1 which had queries that had a completion time between 100 and 300 ms.
% When a query is requested, we randomly pick a relevant query and dispatch it
% according to the trace. The performance metric of interest is  query latency.


% \wlheader{Prediction serving:}
% In prediction serving~\cite{crankshaw2017clipper}, a job processes arriving queries
% to output a prediction, usually obtained via a machine learning model.
% In our set up, we use a random forest regressor as the model and the the news popularity
% dataset~\cite{fernandes2015proactive} for training and test queries in a 50:50 split.
% % Half of the dataset is used to train the model (before the experiment),
% % while the remaining half (test set) is used to generate queries for the workload.
% Queries are picked randomly from the test set and issued in batches of 4.
% The metric of interest is the serving latency.
% % When a query is 


% \wlheader{ML training:}
% % Many real world machine learning deployments require that we continuously update the model
% % from an input stream of data
% % so as to account for changes in the data distribution~\cite{gandiva}.
% % \kkcomment{A good citation here.}
% We use CPUs to train a neural network with four hidden layers of size 64 each.
% We train our model on the naval propulsion~\cite{coraddu2016machine} dataset using
% stochastic gradient descent (SGD).
% Each task in this workload consists of training a batch of 16 points for 100 iterations. The performance metric of interest here is the batch throughput.

\textbf{Traces.} Queries to the database and prediction serving workloads
 are dispatched by a trace-driven workload generator.
We
use the Twitter API~\cite{twitter2020} to collect a trace of tweet arrival rates at Twitter's Asia datacenters;
to bring to parity with our cluster, we subsample the arrival rate by a factor
of 10. The query arrival rate of this trace is visualized in Figure \ref{fig:twitter_trace}.
% for the database workloads and by a factor of 2 for the prediction serving workload.
% The workload generator parses this trace and dispatches a query at every arrival.
For the ML training workload, we draw queries from an essentially infinite pool to create a constant
stream of work.
\insertFigTwitterTrace
% All these workloads are made to report their performance to the cilantro client
% every 10 seconds. Tasks which are running when the performance is reported are discarded to avoid
% affecting results for next reporting round.


% We have shown this trace in Fig.~\ref{fig:twitter_trace}, Appendix~\ref{sec:app_evaluation}.
% For the database workloads, it chooses a random query from the bin while for the query-serving
% workload, it dispatches a batch of 32 data points from the test dataset.

%  and by running the other workloads
% in parallel to replicate its behaviour during our experiments.
% In particular, we found that profiling them by running them independently did not reflect the
% behavior of the workload under interference.
% \kkcomment{@Romil: should we mention this or is it dicey?}


% \subsubsection{Experimental Set up}
\textbf{Experimental set up.}
We use a cluster of  250 AWS m5.xlarge instances (4 vCPUs each).
The \cilantro{} scheduler runs on its own dedicated m5.xlarge instance.
We use the above 4 workloads to create 20 jobs as follows:
10 database jobs with P90, P90, P90, P90, P95, P95, P95, P95, P99, P99 latency SLOs of 2s;
% 3 DB-0 users with P90, P90, and P95 latency SLOs of 2s;
% 7 DB-1 users with P90, P90, P95, P95, P95, P99, P99 latency SLOs of 2s;
3 prediction serving jobs  with P90, P90, and P95 latency SLOs of 2s;
7 ML training jobs  with throughput SLOs of 400, 400, 450, 450, 500, 500, and
500 QPS.
% For each jobs, to discount utility with SLO violations,
% we choose one of the three options in Fig.~\ref{fig:utilityillus},
% to reflect settings where small SLO violations may be either critical or inconsequential.
To reflect settings where small SLO violations may be either critical or inconsequential,
we discount the utility via one of the three options in Fig.~\ref{fig:utilityillus} for each job.
% Detailed information on the users' jobs is given in the appendix.
The estimated total amount of resources based on the median demand was 1637 CPUs;
hence, even at full capacity, not all users can satisfy their SLOs.
We evaluate all baselines for 6 hours.



\subsubsection{Baselines}
\label{sec:baselinesfixedclus}
\vspace{-1mm}

\newcommand{\bstitle}[1]{\textbf{#1.}}
\bstitle{Oracular policies}
We implement the three policies
in~\S\ref{sec:oracularpolicies} with oracular
access to the true performance mappings (obtained by exhaustively profiling workloads for at least 4 hours).
They are \oraclesw, \oracleew,  for maximizing social/egalitarian welfare
and the \oraclenjcs fairness policy.
% For \oraclesws and \oracleew, on each round,
% we maximize the welfare computed via the profiled performance curves using an evolutionary
% algorithm.
% Details of this evolutionary algorithm are given in the supplementary material.
% For \oraclenjc, we compute the demands using the profiled performance curves.

\bstitle{\cilantros policies}
We evaluate \cilantrosw, \cilantroew, and \cilantronjc, as described in
Sec.~\ref{sec:learningpolicies}.
% using \cilantro's online
% learning framework as described in Sec.~\ref{sec:learningpolicies}. 
% For \cilantrosws and \cilantroew, on each round,
% we maximize an upper confidence bound (UCB) on the welfare computed via the UCBs
% on the performance curves using an evolutionary algorithm.
% Since the UCBs are available in memory analytically this can be done efficiently.
% For \cilantronjc, we use the procedure described in~\citet{kandasamy20online}.

\bstitle{Other heuristics}
We implement four methods for fairness and
maximizing  welfare.
While not based directly off specific prior work, such methods are common in the
scheduling literature~\cite{crankshaw2017clipper,grandl2016altruistic}. \equalshare~simply
allocates an equal amount of resources to each job. \evoalgsw~and \evoalgew~are evolutionary
algorithms for social and egalitarian welfare; the same procedure used for Cilantro's welfare
policies, but now operating directly on the performance metrics.
\greedyew~starts by
allocating resources equally;
on each round, it evaluates job utilities in the previous round and
takes away one CPU each from the top half of the users who had high utility 
and allocates it to the bottom half.
% Then it chooses the $k$ jobs with the highest utility, takes away one resource each from these
% jobs, and then allocates it to the $k$ jobs with the lowest utility.
% In our evaluation, we used $k=3$; for smaller values of $k$, the changes were small
% and it resulted in slow convergence while for larger values the changes resulted in large
% fluctuations.

\bstitle{Baselines from prior work}
% To the best of our knowledge, there are no prior methods for the above criteria
% when the performance is unknown.
We adapt five feedback-driven methods from prior work to compare against Cilantro - \ernest~\cite{venkataraman2016ernest},
\quasar~\cite{delimitrou2014quasar},  \minerva~\cite{nathan2019end},
\parties~\cite{chen2019parties}~and \AIMD{}
(Multiplicative-Increase/Additive-Decrease)~\cite{chiu1989analysis}.

\begin{enumerate}[label=\arabic*)]

    \item \ernest~\cite{venkataraman2016ernest}:
    % Ernest uses a featurized linear model to estimate the time taken to run a job as a function of the
    % amount of resources allocated for the job and the load.
    % Equipped with this estimate, we approximate the resource demand for a job, i.e. the amount of
    % resources required, to satisfy a given target SLO for a given load.
    % On each round, we use the demand estimated in the above fashion as inputs to the max-min fairness
    % procedure to compute allocations for all jobs.
    % In our implementation, the Cilantro learners maintain this featurized linear model, which is
    % updated using non-negative least squares regression (as suggested by the authors) whenever there is
    % new data.
    Ernest uses a featurized linear model  to estimate the
    time taken to run a job.
    We use this estimate to approximate the resource demand to meet the job's SLO.
    On each round, we use the estimated demand as inputs to NJC to compute the allocations.
    % In our implementation, the Cilantro learners maintain this featurized linear model, which is
    % updated using non-negative least squares regression (as suggested by the authors) whenever there is
    % new data.
    \vspace{-0.05in}
    
    \item \quasar~\cite{delimitrou2014quasar}:
    Quasar uses collaborative filtering to estimate a job's resource demand, which we
    % We use this estimated deand as inputs to NJC to compute the allocations.
    use as inputs to NJC to compute the allocations.
    % The authors also describe procedures for vertical scaling and co-locating workloads based on
    % interference and heterogeneity.
    We do not incorporate mechanisms for vertical scaling and workload co-location described
    in~\cite{delimitrou2014quasar} to be consistent across all methods.
    % and since our experimental platform does not support extracting these metrics.
    \vspace{-0.05in}
    
    \item \minerva~\cite{nathan2019end}: 
    Minerva sets the allocation for
    job $j$ at each step to be proportional to $\allocj/\utilj$ where $\allocj$ and $\utilj$ are the
    allocation and utility at the previous round.
    % The stated goal of Minerva is to maximize the egalitarian welfare.
    % The authors also propose several video-streaming specific optimizations in addition to this core
    % algorithm which are not applicable in our setting.
    % The authors also describe an experiment design procedure for choosing an initial data set for
    % estimating the model; however, following this procedure for all jobs in our set up resulted in
    % infeasible allocations.
    % Hence, in the first 10 rounds of allocations, we chose the allocations randomly to initialise the
    % models for each job.
    
    \item \parties~\cite{chen2019parties}: 
    Parties upsizes the allocation for a job
    if it violates or is close to violating the SLO, downsizes the allocation if the job comfortably
    satisfies the SLO, and otherwise does nothing.
    If the SLOs of all jobs cannot be met, it evicts the job from the server.
    As eviction is not an option in our setting %as we we wish to fairly allocate resources,
    we use the Parties logic to compute the demands which are then fed to NJC to obtain the allocations.
    For upsizing, we increase the demand by 20 CPUs and for downsizing, we decrease it by 5.
    These parameters were tuned so that the policy did reasonably well on all three metrics.  \cameratext{In particular, we note that applying the \parties~notion of migration in our setting would imply moving the job to a different cluster or increasing the size of the cluster, both of which are beyond scope for this fixed cluster setting.}
    
    \item \AIMD{} (Multiplicative-Increase/Additive-Decrease)~\cite{chiu1989analysis}: 
    This is inspired by TCP congestion control.
    If a user's job violates the SLO, we increase its demand by $1.5\times$ the current allocation,
    and if it satisfies the SLO, we set the demand to be one minus the current allocation.
    We then invoke NJC to compute the allocation for the next round.
    These parameters were tuned so that the policy did reasonably well on all three metrics.
    \end{enumerate}


% In addition to the above three, we also considered the following prior work which could not be
% included in our evaluation.
% Google Autopilot's~\cite{rzadca2020autopilot} horizontal scaling procedure requires an application
% developer to specify a target CPU utilization and computes the resource demand based on the given
% target and the observed utilization.
% This requires an application developer to profile their workload to compute this target based on
% their SLO, which can be laborious.
% Additionally, our experimental platform does not support extracting CPU utilization metrics.
% Paragon~\cite{delimitrou2013paragon} and Allox~\cite{le2020allox} study the placement of jobs
% among heterogeneous resources and have no natural adaptation to our setting.
% Festive~\cite{jiang2012improving} describes a method for fair video streaming based on utilities,
% but their method cannot be adapted to data center resource allocation.







\subsubsection{Results \& Discussion}
\label{sec:baselines}

\insertFigMetrics

\insertFigTimeseriesUtilities
\textbf{Evaluation on performance-aware fairness metrics.}
We first compare all 15 baselines on the social
welfare~\eqref{eqn:socwel}, egalitarian welfare~\eqref{eqn:egalwel}, and the NJC fairness
criteria~\eqref{eqn:njcfair}.
% 
Fig.~\ref{fig:metrics} illustrates the results by
plotting the time-averaged NJC fairness vs the two welfare criteria.
Table~\ref{tab:metrics} tabulates these values explicitly with error bars.
While the oracular methods perform best on their respective metrics, we find that the online
learning policies in \cilantros come close to matching them.
\equalshare{} achieves a perfect NJC score by definition, but performs poorly on social and
egalitarian welfare as it is performance oblivious.

We found that
\greedyew, \parties, and \AIMD{} were sensitive to the amount
by which we changed the allocations based on feedback;
when tuning them, we found that they were
either too slow or too aggressive when responding to load shifts.
Next, the learning models used by \quasar{} and \ernest{} were not able to
accurately estimate the demands in our experiment.
Finally,  the evolutionary baselines were inefficient,
taking a long time to discover the optimal solution.
They, however, were effective within Cilantro's welfare policies
when you need to optimize a cheap analytically computable function as they can be run for
several iterations.
% they took a long time to discover the optimal solution when they 

% the evolutionary baselines \evoalgsws and \evoalgews took a
% long time to discover the
% optimal solution and could not account for fluctuations in the load. \rbcomment{Should we plot these on the timeseries too?}
% They however, were effective when used in our social/egalitarian welfare baselines as they are
% optimizing a cheap analytically computable function available in memory; hence, they can be run for
% several iterations to compute the next allocation.
% executed for several iterations for 
% 
% and hence did not do well uniformly on all criteria.


\insertFigUserUtilities

% We see that,
Despite our general approach, \cilantro's policies are able to
outperform \minerva{} and \greedyew{} which are designed specifically to maximize egalitarian
welfare. It also outperforms generically designed evolutionary algorithms for the social and
egalitarian welfare.
While it may indeed be possible to design more efficient fine-tuned policies for a given objective,
the flexibility provided by Cilantro's approach is beneficial to end users.
It should not be surprising that
\cilantros outperforms other systems such as \ernest, \quasar, \parties, and \AIMD{}
as our policies are designed to explicitly optimize for these objectives.
% this should not be surprising as our policies were designed to explicitly optimize for these metrics.
\emph{But this is precisely the goal of Cilantro}.
End-users can declare their desired objective, and Cilantro will automatically derive
policies to achieve them. 


To illustrate how Cilantro improves with feedback,
in
Fig.~\ref{fig:timeseriesutilities}, we have shown how the three objectives evolve over time for
Cilantro's policies.
% Cilantro. For each of the three welfare criteria, the corresponding Cilantro objective maximizes it
% without requiring an explicit policy.
\equalshare~trivially achieves $\njcfair=1$ at start since our initial allocation
is always 50 CPUs to each job (i.e \equalshare). However, it does poorly on welfare due to poor cluster usage.
The goal behind \cilantronjcs is to achieve $\njcfair=1$ while also achieving good cluster usage.
This causes the initial drop in performance for \cilantronjc~as it explores better allocations that still maximize $\njcfair$.

%\rbcomment{Should we add something more here?}
% For \cilantronjcs this equal allocation continues for a few more rounds 
% In NJC Fairness, <explain why cilantro-NJC drops at start>. We
% note that the convergence of Cilantro can be sped up by using more aggressive allocation windows at
% the cost of slightly reduced accuracy.

\cameratext{Table \ref{tab:metrics} presents the detailed results of our multi-tenant cluster resource sharing evaluation. This table adds a  metric which measures the useful resource usage.}
\begin{align}
\text{Useful resource usage}
= \sum\nolimits_{j=1}^m \min(\allocj, d_j)
\label{eqn:effresusage}
\end{align}
\cameratext{Here, the  $d_j$ is user $j$'s resource demand.
% clipped (see Fig.~\ref{fig:utilityillus}).
This demand-based metric,
measures how much \emph{useful} work is being done by the cluster as allocations beyond the
demand do not increase a user's utility (see Fig.~\ref{fig:utilityillus}). We find that Cilantro's policies achieve the maximum useful resource usage in their respective classes. This is because learning resource demands allows Cilantro to reallocate resources from jobs which have already achieved maximum utility to jobs which can benefit from increased resources. }

\insertTableMetrics

\insertFigSP
% \insertFigHotelReservation
\insertFigMSResults

\textbf{Individual user utilities.}
To delve deeper into the trade-offs of the three paradigms discussed in
\S\ref{sec:polfc}, we have shown the individual  user utilities achieved by these three policies
in Fig.~\ref{fig:userutils}.
We see that both the social and egalitarian welfare policies result in some users being worse off
than receiving their fair allocation of $1000/20=50$ CPUs.
This results in an NJC fairness violation.
In contrast, in \cilantronjc, users are at most marginally worse off than their fair share.
However, a third of the users achieve a noticeably higher utility than their fair share utility,
with more than $3\times$ for a few of them.
% \cilantronjc{} is able to learn the resource demands of all users, and is able to achieve a fair and
% efficient allocation by transfering resources from users whose fair share is in excess of their
% demand to the rest.
We also see that \cilantroews has  maximized egalitarian welfare by taking
resources away from those who achieve high utility and giving it to those who do not, while
\cilantrosws has maximized  social welfare by allocating more resources to jobs that
can quickly achieve high utility.




\textbf{Evaluating Strategy-proofness.}
We next evaluate Cilantro policies for strategy-proofness.
% In our setting,
A policy is said to be strategy-proof if an unscrupulous user cannot increase the
utility of their job by misreporting their performance metrics to the scheduling
policy.
% In some use cases, strategy-proofness can be an important consideration.
% If all users overstate their resource requirements, then a scheduling policy can nevery accurately
% know the resource requirement and hence will not be able to achieve the desired goal.
% % Strategy-proofness
% Most fair division policies are known to be strategy-proof when the utilities are known and hence it
% is natural to ask if this is also true when the utilities need to be learned.
% 
For this, we repeat the same experiment set up; all jobs behave exactly as before except the \incmtt{db16} job which lies about its performance
by either under-reporting by a factor $\times 1/2$,
or over-reporting  by a factor $\times 2$.
% \begin{enumerate}
% \item It states its latency SLO to be P99, 2s (instead of P95).
% \vspace{-0.05in}
% \item When it sends performance updates to the scheduler
% it under-reports its performance by a factor $\times 1/2$.
% \vspace{-0.05in}
% \item It states its latency SLO to be P75, 2s.
% \vspace{-0.05in}
% \item It over-reports its performance  by a factor $\times 2$.
% % value of its observed performance.
% \end{enumerate}
By under-reporting, the user
gives the impression that more resources are required to reach its SLO;
% performance goals.
in contrast, by over-reporting, a user is deceiving the scheduler to
prioritize their job as
they can achieve high utility with few resources.
% in the latter two, the job presents an overly optimistic picture of its resource
% requirements.
% 
In Fig.~\ref{fig:spness},
we report the utilities achieved by \incmtt{db16} under these untruthful behaviors.
%  for \cilantrosw, \cilantroew, and \cilantronjc.
We see that for \cilantronjc, the job's utility does not increase when over-reporting
and decreases when underreporting, leaving no incentive for the user to be untruthful.
In contrast,  for \cilantroew, a user stands to gain by under-reporting
while for \cilantrosw, they gain by over-reporting.
While a theoretical study of such strategy-proofness properties is beyond the scope of this work,
it is interesting to empirically observe that the strategy-proofness properties of NJC fairness
policies are retained in Cilantro.%'s feedback driven setting.
% This is 

% Finally, 
% it is worth pointing out that \cilantronjcs achieves an favourable trade-off between all
% three performance-aware criterion in Fig.~\ref{fig:metrics}.
% With the additional benefit of strategy-proofness,
% this makes \cilantronjcs an attractive option in instances when the overall goals
% of resource allocation cannot be explicitly stated.

%  properties of NJC fair sharing
% mechanisms are retained even when the performance needs to be learned online.
% This makes NJC an attractive option, especially in instances when the overall goals
% of resource allocation cannot be explicitly stated.
% 
% It is interesting to note that the strategy-proofness properties of NJC fair sharing
% mechanisms are preserved even when the resource-performance mappings need to be learned online.
% The strategy-proofness properties of these three policies are not
% Indeed, when utilities are known neither social nor egalitarian policies are known to be
% strategy-proof.
% Indeed, for both \cilantrosws and \cilantroews are not strategy-proof as the achieved utility can be
% manipulated by misreporting.
% This is not surprising since even the oracular versions of these policies are not known to be
% strategy-proof.



