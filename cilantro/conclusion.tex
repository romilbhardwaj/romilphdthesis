

\section{Conclusion}
\label{sec:cilantro_conclusion}
%\vspace{-0.05in}

We described \cilantro{}, a performance-aware framework for the allocation of a finite amount of resources among competing jobs.
% in
% There are two fundamental tenets that underlie  this work.
Our motivations were:
\emph{(i)} resource allocation policies should be performance-aware and based on
real-time feedback in production environments,
% \emph{(ii)} since users' resource-to-performance mappings are difficult and
% burdensome to estimate via offline profiling, they should be done in an online fashion;
\emph{(ii)} schedulers should accommodate diverse allocation objectives.
We designed \cilantros to address these challenges by decoupling the performance learning from the policies and informing the policies of uncertainties in performance estimates, thus
enabling the realization of several performance-aware
policies in multi-tenant and microservices settings.
% such as social/egalitarian welfare or NJC fair division,
% \emph{(iii)} as user's utilities are usually unknown, they need to be estimated from feedback
% on job performance.
% % and performance goals.
% We designed \hmmfs and \hmmfls under this tenets, which provably satisfies PE, fairness, and
% SP.
% \cilantro's policies are competitive with oracular polices which know workload characteristics apriori and outperforms other baselines in fixed cluster sharing and microservice environments.
% Empirically, we demonstrated that it outperforms other fair allocation methods.

